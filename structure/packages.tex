\usepackage[T1]{fontenc}
\usepackage{lmodern}
\usepackage{etoolbox}
\iftagged{BR}{
    \usepackage[brazil]{babel}
}{
    \usepackage{babel}
}
\usepackage[utf8]{inputenc}
\usepackage{graphicx}
\usepackage{lastpage}
\usepackage{xcolor}
\usepackage[colorlinks=true,citecolor=JungleGreen,
linkcolor=NavyBlue,urlcolor=DarkRed,filecolor=green]{hyperref}

\usepackage{tabularx} % Tabelas com espessura definida

\usepackage{tikz,tikz-cd}
\usetikzlibrary{patterns,shadows.blur,cd,babel}
\usepackage{amsmath, amsfonts, amsthm, amssymb, mathtools}
\usepackage{fontawesome5}
\usepackage{makeidx}
\usepackage{pdfpages}
\usepackage{graphicx, wrapfig}
\usepackage{multicol, multirow}
\usepackage{enumerate}
\usepackage[shortlabels]{enumitem}

\newlist{alter}{enumerate}{1}
\SetLabelAlign{Center}{\hfil#1\hfil}
\setlist[alter,1]{label=(\textit{\alph*}), align=Center}

%\renewlist{enumerate}{enumerate}{1}
%\SetLabelAlign{Center}{\hfil#1}
%\setlist[enumerate,1]{label=\text{\arabic*.}, align=Center}

\newlist{itroman}{enumerate}{1}
\setlist[itroman,1]{label=\ensuremath{(\roman*)}}

\makeatletter
\def\namedlabel#1#2{\begingroup
    #2%
    \def\@currentlabel{#2}%
    \phantomsection\label{#1}\endgroup
}
\makeatother

%\usepackage{esint}
\usepackage[makeroom]{cancel}
\usepackage{caption}
\usepackage{subcaption}
\usepackage{xfrac}
\usepackage{pgfplots}
\pgfplotsset{compat=1.16}
\usepackage{marvosym}
\usepackage{float}
\usetikzlibrary{calc}
%\usepackage{wasysym}
%\usepackage{arydshln}
\usepackage{appendix}
\usepackage{stmaryrd}
\usepackage{mathrsfs}
\usepackage{natbib}
\usepackage[new]{old-arrows}

% mathscr bold
\usepackage{mathrsfs}
\DeclareMathAlphabet{\mathscrbf}{OMS}{mdugm}{b}{n}

\usepackage{fancyvrb}
\usepackage{listings}


\usepackage{centernot}

% Use for 0 no prefix. Redefine as needed per-question.
\newcommand{\writtensection}{0}
%--------------------------------------

%Hyphenation rules
%--------------------------------------
\usepackage{hyphenat,hyperref}
%\hyphenation{mate-mática recu-perar}


\setlist{  
	listparindent=\parindent,
	%	parsep=0pt,
}

\usepackage{url}
\usepackage{lipsum}
\usepackage{epigraph}
\usepackage{transparent}
%\usepackage[text={}]{draftwatermark}

\usepackage{graphicx}
\usepackage{mathtools}

%%%%%%%%%%%% Colors
\definecolor{airforceblue}{rgb}{0.36, 0.54, 0.66}
\definecolor{armygreen}{rgb}{0.29, 0.33, 0.13}
\definecolor{caribbeangreen}{rgb}{0.0, 0.8, 0.6}
\definecolor{amber(sae/ece)}{rgb}{1.0, 0.49, 0.0}
\definecolor{aquamarine}{rgb}{0.5, 1.0, 0.83}
\definecolor{violet}{rgb}{0.56, 0.0, 1.0}


\definecolor{arsenic}{rgb}{0.19,0.19,0.19}
\hypersetup{
	linkcolor  = blue!75!black,
	citecolor  = blue!75!black,
	urlcolor   = violet!45!black,
	colorlinks = true,
}

\newtheoremstyle{BRstyle}%
{4.5pt}% Space above
{}% Space below 
{}% Body font
{}% Indent amount
{\bfseries}% Theorem head font
{.}% Punctuation after theorem head
{.4em}% Space after theorem head 
{}% Theorem head spec (can be left empty, meaning ‘normal’) %{{\underline{\thmname{#1}\thmnumber{ #2}~\thmnote{(#3)}}}}

\usepackage[framemethod=TikZ]{mdframed}

\surroundwithmdframed[
  topline=false,
  rightline=false,
  bottomline=false,
  leftmargin=\parindent,
  skipabove=\medskipamount,
  skipbelow=\medskipamount
]{invocacao}

\usepackage{subfiles}

\usepackage{scalerel}

% Corrige a fonte caligráfica
\usepackage{euscript}
%\renewcommand{\mathcal}{\EuScript}

\usepackage{setspace}

\theoremstyle{BRstyle}
\tagged{BR}{
	\usepackage[brazil]{babel}
	\usetag{BR}
}

\newtheorem{teorema}{\iftagged{BR}{Teorema}{Theorem}}[section]
\newtheorem{teoremas}[teorema]{\iftagged{BR}{Theorems}{Theorems}}
\newtheorem{definicao}[teorema]{\iftagged{BR}{Definição}{Definition}}
\newtheorem{definicoes}[teorema]{\iftagged{BR}{Definições}{Definitions}}
\newtheorem{conjectura}[teorema]{\iftagged{BR}{Conjectura}{Conjecture}}
\newtheorem{proposicao}[teorema]{\iftagged{BR}{Proposição}{Proposition}}
\newtheorem{proposicoes}[teorema]{\iftagged{BR}{Proposições}{Propositions}}
\newtheorem{corolario}[teorema]{\iftagged{BR}{Corolário}{Corollary}}
\newtheorem{corolarios}[teorema]{\iftagged{BR}{Corolários}{Corollarys}}
\newtheorem{axioma}[teorema]{\iftagged{BR}{Axioma}{Axiom}}
\newtheorem{lema}[teorema]{\iftagged{BR}{Lema}{Lemma}}
\newtheorem{lemas}[teorema]{\iftagged{BR}{Lemas}{Lemmas}}
\newtheorem{afirmacao}[teorema]{\iftagged{BR}{Afirmação}{Afirmation}} 

\makeatletter
    \def\@endtheorem{\hfill\faCoffee \endtrivlist\@endpefalse }
\makeatother
\newtheorem{exemplo}[teorema]{\iftagged{BR}{Exemplo}{Example}}
\newtheorem{exemplos}[teorema]{\iftagged{BR}{Exemplos}{Examples}}
\newtheorem{construcao}[teorema]{\iftagged{BR}{Construção}{Construction}}
\newtheorem{observacao}[teorema]{\textit{\iftagged{BR}{Observação}{Remark}}}
\newtheorem{notacao}[teorema]{\iftagged{BR}{Notação}{Notation}}
\newtheorem{invocacao}[teorema]{\iftagged{BR}{Invocação}{Summoning}}
\newtheorem{contraexemplo}[teorema]{\iftagged{BR}{Contra-exemplo}{Counterexample}}
\newtheorem{exercicio}[teorema]{\iftagged{BR}{Exercício}{Exercise}}
\newtheorem{exercicios}[teorema]{\iftagged{BR}{Exercícios}{Exercises}}
\usepackage{catchfilebetweentags}

\usepackage{setspace}

% Cabeçalhos e rodapés antes do corpo do texto
\usepackage{fancyhdr}
\renewcommand*{\headrulewidth}{0pt}
\fancyhf{}
\pagestyle{fancy}

\newcommand{\acertacabecalhos}[0]{
  \fancyhf[lh,rh,ch]{}
  \fancyhf[leh]{\slshape \nouppercase{\leftmark}}
  \fancyhf[roh]{\slshape \nouppercase{\rightmark}}
  \fancyfoot[C]{\thepage}
  \renewcommand*{\headrulewidth}{0.4pt}
  \renewcommand*{\footrulewidth}{0pt}

  \setlength{\headheight}{13.59999pt}
}
\acertacabecalhos